\chapter{Introdução} \label{cap:metod}

   Com o avanço da tecnologia em plataforma \textit{web} para dispositivos móveis, o número de usuários utilizando aplicativos tem crescido exponencialmente. Segundo a \citeonline{} (citar App Annie - State of Mobile), o Brasil é o terceiro país em que usuários passam mais tempo em aplicativos, com uma média de 3 horas e 40 minutos por usuário %POR DIA, POR MÊS? ESPECÍFIQUE CORRETAMENTE%, com uma população de 2 \textit{smartphones} para cada cidadão brasileiro. \cite{} (citar fundação Getúlio Vargas) %AS CITAÇÕES JÁ DEVEM ESTAR TODAS CORRETAS NO DOCUMENTO%
   
   Grande parte do uso desses aplicativos são desenvolvidos para serem instalados em dispositivos móveis e executados nativamente em plataformas Android ou IOs, e a grande maioria são aplicativos de empresas bem estabelecidas, como Facebook, Uber e Netflix. \cite{} (Citar APP Annie). Contudo, quando se trata de aplicações pequenas, há possibilidade de que o aplicativo nunca seja baixado - de acordo com a \citeonline{} (citar comScore), a média de de Apps sem downloads \textit{smarthphones} é de 51\%. Um dos grandes motivos, refere-se a falta de espaço no armazenamento do dispositivo e a necessidade de procurar os aplicativos em \textit{App Stores}. % A ESCRITA DESTE PARÁGRAFO FICOU RUIM, MELHORE%
   
   A publicação da TIC 2018 (citar CATIC ou TIC?), mostra que a  maioria da população brasileira, tanto em zonas urbanas quanto rurais, possui conexões de internet abaixo de 4 Mb/s, comparado a média nacional brasileira de [checar valor mais recente]. O uso de \textit{Progressive Web Apps} (PWAs), busca eliminar %OU MINIMIZAR,OU VIABILIZAR A REDUÇAO. NUNCA PROMETA O QUE NÃO DEPENDE DE VOCE E O QUE ÉIMPOSSÍVEL% esses problemas, eliminando barreiras para que usuários sem poder aquisitivo possam utilizar aplicativos 
   
   PWAs são aplicativos que funcionam diretamente em um navegador, seja este em dispositivo móvel ou \textit{desktop}, facilitando a instalação no dispositivo no qual o usuário esteja utilizando, bastando apenas adicionar a página na tela principal, quando necessário instalar em um dispositivo móvel. \cite{} (Citar Lepage - Seu Primeira PWA). Com isso, não há necessidade de utilizar o armazenamento de dados do dispositivo móvel para instalação de um aplicativo. %REVER AQUI TAMBÉM A ESCRITA%
   
   Contudo, o uso de PWAs ainda é menor em relação ao uso de aplicativos nativos, onde usuários gastam mais tempo em seus dispositivos (citar dados). Com isso, surge a dependência de pesquisas sobre a nova tecnologia, citando o potencial de PWAs, comparadas à de aplicativos nativos. (Citar documentação)
   
   O objetivo principal deste trabalho é realizar testes de software em diferentes sistemas operacionais e situações exteriores %COMO ASSIM EXTERIORES?% - como localização e velocidade de internet - e obter resultados quantitativos que possam ser comparados para demonstrar as diferenças de desempenho e qualidade de aplicativos nativos e PWAs.
   
   \section{OBJETIVOS}
   
   Nesta seção, será descrito o objetivo geral do trabalho, dividido em objetivos específicos.
   
   \subsection{OBJETIVO GERAL}
   
   O objetivo geral deste trabalho é de avaliar o desempenho de aplicativos PWAs em diversas %TALVEZ SEJA MELHOR LIMITAR% plataformas móveis e comparar os resultados com seus respectivos aplicativos nativos equivalentes, avaliando a sua velocidade de resposta, teste de estresse, uso de armazenamento de dados, entre outras métricas quantitativas.
   
   \subsection{OBJETIVO ESPECÍFICOS}
   
   - Agrupar aplicativos nativos e PWAs para avaliação; %NÃO ENTENDI%
   
   - Realizar diversos %QUAIS?% testes de \textit{software} a fim de coletar os dados necessários para avaliação;
   
   - Comparar %COMO?% resultados entre os aplicativos nativos e PWAs, a fim de determinar a melhor opção para casos específicos.
   
   %PENSO QUE ESTES OBJETIVOS ESTEJAM MUITO EM BAIXO NIVEL%
   
   \subsection{JUSTIFICATIVA}
   
   PWA é um termo relativamente novo, com a justificativa de que, aplicativos distribuídos em \textit{web} possam ser utilizados à nível similar de aplicativos nativos, por meio do navegador, possibilitando o uso dos mesmos, independente de qualidade de internet (citar LEPAGE). Ainda não existem pesquisas com dados quantitativos que sustentem essa ideia. Levanta-se então um estudo para avaliação de uso desses aplicativos, por meio de ferramentas de testes de software, a fim de comparar e obter resultados para analise. %SERÁ QUE ISSO QUE ESCREVEU É JUSTIFICATIVA?%