
\documentclass[oneside]{normas-utf-tex} %oneside = para dissertacoes com numero de paginas menor que 100 (apenas frente da folha) 

% force A4 paper format
%\special{papersize=210mm,297mm}

\usepackage[brazil]{babel} % pacote portugues brasileiro
\usepackage[utf8]{inputenc} % pacote para acentuacao direta
\usepackage{amsmath,amsfonts,amssymb} % pacote matematico
\usepackage{graphicx} % pacote grafico
%\usepackage{times} % fonte times
\usepackage[final]{pdfpages} % adicao da ata
\usepackage{hyperref} % gera hiperlinks para o sumario, links, referencias -- deve vir antes do 'abntcite' 
\usepackage{tabularx}
\usepackage{pgfgantt}
\usepackage{rotating}
\usepackage[alf,abnt-emphasize=bf,bibjustif,recuo=0cm, abnt-etal-cite=2, abnt-etal-list=99]{abntcite} %configuracao correta das referencias bibliograficas.
\usepackage{subfigure}
\usepackage{xtab}
\usepackage{nomencl}
\usepackage{tabularx}
\newcolumntype{Y}{>{\centering\arraybackslash}X}
\usepackage{multirow}
%\usepackage[portuguese,algoruled,longend,linesnumbered]{algorithm2e} %Para linhas numeradas e réguas de identação
\usepackage{algorithm2e}
\usepackage{listings}
% Definindo novas cores
\definecolor{verde}{rgb}{0.25,0.5,0.35}
\definecolor{jpurple}{rgb}{0.5,0,0.35}
% Configurando layout para mostrar codigos Java
\lstset{
  language=Java,
  basicstyle=\ttfamily\small,
  keywordstyle=\color{jpurple}\bfseries,
  stringstyle=\color{red},
  commentstyle=\color{verde},
  morecomment=[s][\color{blue}]{/**}{*/},
  extendedchars=true,
  showspaces=false,
  showstringspaces=false,
  numbers=left,
  numberstyle=\tiny,
  breaklines=true,
  backgroundcolor=\color{cyan!10},
  breakautoindent=true,
  captionpos=b,
  xleftmargin=0pt,
  tabsize=4
}
\renewcommand{\lstlistingname}{Código}
\renewcommand\lstlistlistingname{Lista de Códigos}
\newcommand*{\noaddvspace}{\renewcommand*{\addvspace}[1]{}}

\usepackage{helvet}                                % Usa a fonte Helvetica
%\usepackage{times}                                          % Usa a fonte Times
%\usepackage{palatino}                                      % Usa a fonte Palatino
%\usepackage{lmodern}                                       % Usa a fonte Latin Modern

% ---------- Preambulo ----------
\instituicao{Universidade Tecnol\'ogica Federal do Paran\'a} % nome da instituicao
\departamento{Departamento Acad\^{e}mico de Computa\c{c}\~ao} % nome do programa
\programa{Curso de Ci\^{e}ncia da Computa\c{c}\~ao} % área ou curso

\documento{Trabalho de Conclusão de Curso}
\nivel{Graduação}
\titulacao{Bacharel} 

\titulo{{ESTUDO COMPARATIVO DE APLICATIVOS NATIVOS E PROGRESSIVE WEB APPS}} % titulo do trabalho em portugues
\title{\MakeUppercase{TITLE}} % titulo do trabalho em ingles

\autor{Jefferson Andrew de Oliveira} % autor do trabalho
\cita{OLIVEIRA, Jefferson} % sobrenome (maiusculas), nome do autor do trabalho

\palavraschave{Microserviços, Agricultura 4.0} % palavras-chave do trabalho
\keywords{Microservices, Agriculture 4.0} % palavras-chave do trabalho em ingles

\comentario{\UTFPRdocumentodata\ apresentado ao \UTFPRdepartamentodata\ da \ABNTinstituicaodata\ como requisito parcial para obten\c{c}\~ao do título de ``\UTFPRtitulacaodata\ em Computação''.}

\orientador{Prof. Ricardo Sobjak} % nome do orientador do trabalho

\primeiroassina{Prof. Fulano\\ UTFPR - Câmpus Medianeira}
\segundoassina{Prof. Fulano\\ UTFPR - Câmpus Medianeira}
\terceiroassina{Prof. Fulano\\ UTFPR - Câmpus Medianeira}
\quartoassina{Prof. Fulano\\ UTFPR - Câmpus Medianeira}
\textoaprovacao{Este \UTFPRdocumentodata\ foi apresentado \`{a}s xx:xxh do dia X de mês de 20XX como requisito parcial para a obtenção do título de \UTFPRtitulacaodata~no \UTFPRprogramadata, da \ABNTinstituicaodata, Câmpus Medianeira. O candidato foi arguido pela Banca Examinadora composta pelos professores abaixo assinados. Após deliberação, a Banca Examinadora considerou o trabalho aprovado.}

\local{Medianeira} % cidade
\data{\the\year} % ano automatico

% desativa hifenizacao mantendo o texto justificado.
% thanks to Emilio C. G. Wille
\tolerance=1
\emergencystretch=\maxdimen
\hyphenpenalty=10000
\hbadness=10000
\sloppy

\definecolor{laranjautfpr}{cmyk}{0.0, 0.2, 1.0, 0.0}
\usepackage{enumitem}
\setlist{leftmargin=2cm}
\setlist{nosep}


%---------- Inicio do Documento ----------
\begin{document}


\capa % geracao automatica da capa
\folhaderosto % geracao automatica da folha de rosto

\termodeaprovacao

%resumo
\begin{resumo}

Aplicativos Web Progressivos (\textit{Progressive Web Apps - PWA}) tem ganhando espaço na área de desenvolvimento para dispositivos móveis, possibilitando o uso de aplicativos através de um navegador de internet, sem requerer instalações em aparelhos, e ainda ter funcionamento similar à de aplicativos nativos. Por ser uma tecnologia relativamente nova, surge a incerteza de qual tipo de arquitetura utilizar para novos projetos, considerando fatores externos que influenciam no desempenho de aplicativos nativos e PWAs. Este trabalho \textit{apresenta um estudo comparativo entre PWAs e  aplicativos nativos, utilizando testes de caso e demonstrando dados quantitativos, como velocidade de resposta em ambientes diversificados, quantidade de linha de códigos, utilização de armazenamento de dados e engajamento do usuário. Informações geográficas de usuários, velocidade de internet disponível e especificações de sistema de aparelhos móveis serão considerados para avaliação de desempenho de aplicativos nativos e PWAs. Serão utilizados ferramentas de teste de aplicativos móveis para extração dessas informações e por fim, apresentar uma solução para facilitar a decisão de qual tipo de arquitetura utilizar para desenvolvimento de novas aplicações móveis.}

Palavras-chave: PWA Nativo Desenvolvimento Desempenho

\end{resumo}

%abstract
\begin{abstract}


\end{abstract}

%------------------------DEDICATÓRIA------------------------------

\begin{dedicatoria}
Dedico este Trabalho de Conclusão de Curso a minha família e amigos e colegas da UTFPR, Medianeira.
\end{dedicatoria}

%-----------------------AGRADECIMENTOS----------------------------
\begin{agradecimentos}
Ninguém ainda.
\end{agradecimentos}
%-----------------------EPÍGRAFE-----------------------------------
\begin{epigrafe}
“Nós só podemos ver um pouco do futuro, mas o suficiente para perceber que há muito a fazer.”
\end{epigrafe}

%Solução para o problema de numeração errada de equações, figuras e tabelas - Contribuições de Leandro Pasa UTFPR-MD 

\listadefiguras % geracao automatica da lista de figuras
\listadetabelas % geracao automatica da lista de tabelas)
 
% para lista de figuras
\makeatletter
\renewcommand{\thefigure}{\@arabic\c@figure}
\makeatother
 
\makeatletter
\@removefromreset{figure}{chapter}
\makeatother
 
 
% para lista de tabelas
\makeatletter
\renewcommand{\thetable}{\@arabic\c@table}
\makeatother
 
\makeatletter
\@removefromreset{table}{chapter}
\makeatother
 
 
% para equações
\makeatletter
\renewcommand{\theequation}{\@arabic\c@equation}
\makeatother
 
\makeatletter
\@removefromreset{equation}{chapter}
\makeatother 

% Para eliminar, nas listas de figuras e tabelas, o espaçamento que aparece entre figuras de capítulos diferentes:

\addtocontents{lof}{\protect\noaddvspace} % para figuras
\addtocontents{lot}{\protect\noaddvspace} % para tabelas

\IfFileExists{/etc/resolv.conf}{}{\listadefiguras} %Geração de lista de figuras e automáticas
\IfFileExists{/etc/resolv.conf}{}{\listadetabelas} %Geração de lista de tabelas automáticas
% listas (opcionais, mas recomenda-se a partir de 5 elementos)
%\listadequadros % geração automatica da lista de quadros
\listadesiglas % geracao automatica da lista de siglas
%\listadesimbolos % geracao automatica da lista de simbolos

% sumario
\sumario % geracao automatica do sumario


%---------- Inicio do Texto ----------
% recomenda-se a escrita de cada capitulo em um arquivo texto separado (exemplo: intro.tex, fund.tex, exper.tex, concl.tex, etc.) e a posterior inclusao dos mesmos no mestre do documento utilizando o comando \input{}, da seguinte forma:

%\setcounter{page}{48}

\setlength{\parskip}{0.0cm}
\chapter{Introdução} \label{cap:metod}

   Com o avanço da tecnologia em plataforma \textit{web} para dispositivos móveis, o número de usuários utilizando aplicativos tem crescido exponencialmente. Segundo a \citeonline{} (citar App Annie - State of Mobile), o Brasil é o terceiro país em que usuários passam mais tempo em aplicativos, com uma média de 3 horas e 40 minutos por usuário %POR DIA, POR MÊS? ESPECÍFIQUE CORRETAMENTE%, com uma população de 2 \textit{smartphones} para cada cidadão brasileiro. \cite{} (citar fundação Getúlio Vargas) %AS CITAÇÕES JÁ DEVEM ESTAR TODAS CORRETAS NO DOCUMENTO%
   
   Grande parte do uso desses aplicativos são desenvolvidos para serem instalados em dispositivos móveis e executados nativamente em plataformas Android ou IOs, e a grande maioria são aplicativos de empresas bem estabelecidas, como Facebook, Uber e Netflix. \cite{} (Citar APP Annie). Contudo, quando se trata de aplicações pequenas, há possibilidade de que o aplicativo nunca seja baixado - de acordo com a \citeonline{} (citar comScore), a média de de Apps sem downloads \textit{smarthphones} é de 51\%. Um dos grandes motivos, refere-se a falta de espaço no armazenamento do dispositivo e a necessidade de procurar os aplicativos em \textit{App Stores}. % A ESCRITA DESTE PARÁGRAFO FICOU RUIM, MELHORE%
   
   A publicação da TIC 2018 (citar CATIC ou TIC?), mostra que a  maioria da população brasileira, tanto em zonas urbanas quanto rurais, possui conexões de internet abaixo de 4 Mb/s, comparado a média nacional brasileira de [checar valor mais recente]. O uso de \textit{Progressive Web Apps} (PWAs), busca eliminar %OU MINIMIZAR,OU VIABILIZAR A REDUÇAO. NUNCA PROMETA O QUE NÃO DEPENDE DE VOCE E O QUE ÉIMPOSSÍVEL% esses problemas, eliminando barreiras para que usuários sem poder aquisitivo possam utilizar aplicativos 
   
   PWAs são aplicativos que funcionam diretamente em um navegador, seja este em dispositivo móvel ou \textit{desktop}, facilitando a instalação no dispositivo no qual o usuário esteja utilizando, bastando apenas adicionar a página na tela principal, quando necessário instalar em um dispositivo móvel. \cite{} (Citar Lepage - Seu Primeira PWA). Com isso, não há necessidade de utilizar o armazenamento de dados do dispositivo móvel para instalação de um aplicativo. %REVER AQUI TAMBÉM A ESCRITA%
   
   Contudo, o uso de PWAs ainda é menor em relação ao uso de aplicativos nativos, onde usuários gastam mais tempo em seus dispositivos (citar dados). Com isso, surge a dependência de pesquisas sobre a nova tecnologia, citando o potencial de PWAs, comparadas à de aplicativos nativos. (Citar documentação)
   
   O objetivo principal deste trabalho é realizar testes de software em diferentes sistemas operacionais e situações exteriores %COMO ASSIM EXTERIORES?% - como localização e velocidade de internet - e obter resultados quantitativos que possam ser comparados para demonstrar as diferenças de desempenho e qualidade de aplicativos nativos e PWAs.
   
   \section{OBJETIVOS}
   
   Nesta seção, será descrito o objetivo geral do trabalho, dividido em objetivos específicos.
   
   \subsection{OBJETIVO GERAL}
   
   O objetivo geral deste trabalho é de avaliar o desempenho de aplicativos PWAs em diversas %TALVEZ SEJA MELHOR LIMITAR% plataformas móveis e comparar os resultados com seus respectivos aplicativos nativos equivalentes, avaliando a sua velocidade de resposta, teste de estresse, uso de armazenamento de dados, entre outras métricas quantitativas.
   
   \subsection{OBJETIVO ESPECÍFICOS}
   
   - Agrupar aplicativos nativos e PWAs para avaliação; %NÃO ENTENDI%
   
   - Realizar diversos %QUAIS?% testes de \textit{software} a fim de coletar os dados necessários para avaliação;
   
   - Comparar %COMO?% resultados entre os aplicativos nativos e PWAs, a fim de determinar a melhor opção para casos específicos.
   
   %PENSO QUE ESTES OBJETIVOS ESTEJAM MUITO EM BAIXO NIVEL%
   
   \subsection{JUSTIFICATIVA}
   
   PWA é um termo relativamente novo, com a justificativa de que, aplicativos distribuídos em \textit{web} possam ser utilizados à nível similar de aplicativos nativos, por meio do navegador, possibilitando o uso dos mesmos, independente de qualidade de internet (citar LEPAGE). Ainda não existem pesquisas com dados quantitativos que sustentem essa ideia. Levanta-se então um estudo para avaliação de uso desses aplicativos, por meio de ferramentas de testes de software, a fim de comparar e obter resultados para analise. %SERÁ QUE ISSO QUE ESCREVEU É JUSTIFICATIVA?%
    \chapter{Fundamentação Teorica} \label{cap:metod}
    
    
\chapter{Materiais e métodos} \label{cap:metod}
\graphicspath{{Imagens/}}


todo
\input{viabi.tex}
\chapter{Conclusões} \label{cap:concl}

%---------- Referencias ----------
\clearpage % this is need for add +1 to pageref of bibstart used in 'ficha catalografica'.
\label{bibstart}
\bibliography{bibliografia} % geracao automatica das referencias a partir do arquivo bibliografia.bib
% \bibliography{library}
\label{bibend}








% --------- Ordenacao Afabetica da Lista de siglas --------
%\textbf{* Observa\c{c}\~oes:} a ordenacao alfabetica da lista de siglas ainda nao eh realizada de forma automatica, porem
% eh possivel se de realizar isto manualmente. Duas formas:
%
% ** Primeira forma)
%    A ordenacao eh feita com o auxilio do comando 'sort', disponivel em qualquer
% sistema Linux e UNIX, e tambem em sistemas Windows se instalado o coreutils (http://gnuwin32.sourceforge.net/packages/coreutils.htm)
% comandos para compilar e ordenar, supondo que seu arquivo se chame 'dissertacao.tex':
%
%      $ latex dissertacao
%      $ bibtex dissertacao && latex dissertacao
%      $ latex dissertacao
%      $ sort dissertacao.lsg > dissertacao.lsg.tmp
%      $ mv dissertacao.lsg.tmp dissertacao.lsg
%      $ latex dissertacao
%      $ dvipdf dissertacao.dvi
%
%
% ** Segunda forma)
%\textbf{Sugest\~ao:} crie outro arquivo .tex para siglas e utilize o comando \sigla{sigla}{descri\c{c}\~ao}.
%Para incluir este arquivo no final do arquivo, utilize o comando \input{arquivo.tex}.
%Assim, Todas as siglas serao geradas na ultima pagina. Entao, devera excluir a ultima pagina da versao final do arquivo
% PDF do seu documento.


%-------- Citacoes ---------
% - Utilize o comando \citeonline{...} para citacoes com o seguinte formato: Autor et al. (2011).
% Este tipo de formato eh utilizado no comeco do paragrafo. P.ex.: \citeonline{autor2011}

% - Utilize o comando \cite{...} para citacoeses no meio ou final do paragrafo. P.ex.: \cite{autor2011}



%-------- Titulos com nomes cientificos (titulo, capitulos e secoes) ----------
% Regra para escrita de nomes cientificos:
% Os nomes devem ser escritos em italico, 
%a primeira letra do primeiro nome deve ser em maiusculo e o restante em minusculo (inclusive a primeira letra do segundo nome).
% VEJA os exemplos abaixo.
% 
% 1) voce nao quer que a secao fique com uppercase (caixa alta) automaticamente:
%\section[nouppercase]{\MakeUppercase{Estudo dos efeitos da radiacao ultravioleta C e TFD em celulas de} {\textit{Saccharomyces boulardii}}
%
% 2) por padrao os cases (maiusculas/minuscula) sao ajustados automaticamente, voce nao precisa usar makeuppercase e afins.
% \section{Introducao} % a introducao sera posta no texto como INTRODUCAO, automaticamente, como a norma indica.


\end{document}
